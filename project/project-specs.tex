\documentclass{hw}

\title{}
\author{MATH 168, Spring 2022}
\date{Course Project}


\begin{document}

\section*{MATH 168 Course Project Description}

The purpose of the final project in MATH 168 is to help you explore a topic that interests you, practice scientific communication, and navigate scholarly research literature. 

Your final project for MATH 168 is worth 50 points toward your final grade, and consists of the following main components: 
\begin{itemize}
    \item A final report on the findings of your project, formatted as a scholarly article (20 points, due June 10th). 
    \item A short in-class presentation (5 points, due June 1st). 
    \item Several short essays, some of which you have already completed (20 points, due throughout the quarter). 
    \begin{itemize}
        \item Short Essay 4 is your project proposal, which you will complete as a group.  
    \end{itemize}     
    \item Several small milestones throughout the quarter in which you will report your progress (5 points, due throughout the quarter). 
\end{itemize}

\subsection*{Project Topic: General Considerations}

In operational terms, a good project topic is one for which the following are true: 
\begin{itemize}
    \item You have a clearly defined \textbf{question}. 
    \item You have a plan for how to \textbf{approach} that question using data analysis, mathematical modeling, simulations, or other tools. 
    \item You have a plan to \textbf{access the resources} that you need. 
    The most common kinds of resource needs are data sets and computational power. 
\end{itemize}
You might be able to check off these items and still not know whether your project idea is ``good'' or not. 
If you're not sure, here's a guideline: your project idea is good if, to the best of your ability to predict, 
\begin{itemize}
    \item You expect to \textbf{learn} some interesting things about networks and the world around you by completing it. 
    \item You expect to be \textbf{proud} of what you create with your group members. 
\end{itemize}

\subsection*{How Will the Project Be Graded?}

Most project components will be graded in a more traditional fashion, with partial credit and no re-attempts. 
There may be some project updates that are incorporated into homework assignments, in which case the homework grading methodology will be used. 

\subsubsection*{Group Contributions}

My expectation is that all project members will contribute approximately equally to all project assignments. 
Therefore, by default, all members of a project group will receive the same grade for assignments submitted as a group. 

If you are experiencing persistent issues in which a group member is declining to contribute to the project, I encourage you to first reach out to them with empathy and understanding. 
Sometimes, there are obstacles to our full participation that can be hard for others to see. 
That said, if you've tried this approach and are still having participation issues, please contact me. 

The final report is required to include a Group Contributions Statement that will outline in some detail the role that each group member played in the project. 

\pagebreak

\subsection*{Expectations for Final Reports}

Your final report is the primary way in which you will communicate the details of your findings. 
The report is expected to emulate the structure of a scholarly article published the high-profile journal \emph{Proceedings of the National Academy of Sciences} (PNAS). 
You will create your report using the same \href{https://www.overleaf.com/latex/templates/template-for-preparing-your-research-report-submission-to-pnas-using-overleaf/fzcbzjvpvnxn}{PNAS \LaTeX{} template} that prospective authors use to submit to the journal. 



\textbf{Your report should be no shorter than $2+k$ pages and no longer than $4+k$ pages}, where $k$ is the number of group members. 
Your report should contain the following sections, in the order specified. 
Exceptions to this structure may be appropriate in certain cases, but must be \textbf{requested in advance}. 
\begin{enumerate}
    \item An \textbf{Abstract} which summarizes your question, methods, and findings in under 100 words.  
    \item An \textbf{Introduction} which highlights your question and describes why it is important. 
    Your introduction should contain a \textbf{literature review} of scholarly work related to your question. 
    You should discuss \textbf{5 sources related to your scientific question} and \textbf{5 sources related to your methodology} that you use to answer your question.  
    \begin{itemize}
        \item Throughout your essay, citations must be handled using the \href{https://www.overleaf.com/learn/latex/Bibliography_management_with_bibtex}{Bib\TeX} citation management system. 
    \end{itemize}
    \item A \textbf{Materials and Methods} section which describes your data source and how you acquired it; any data preparation or cleaning you may have performed; your data analysis or modeling approach and how you designed it; and any other considerations relevant for understanding your approach to the problem. 
    \begin{itemize}
        \item This section is required to include at least \textbf{4 equations} related to your data analysis or modeling techniques.  
    \end{itemize}
    \item A \textbf{Results} section which describes your findings. 
    This section should include and discuss at least $2 + k$  scientific data visualizations or tables, where $k$ is the number of group members. 
    For example, a project with 3 group members should include at least $2+3 = 5$ figures or tables.  
    \begin{itemize}
        \item Data visualizations should have legends, axis labels, and captions describing the experiment or analysis that generated the visualization. 
        \item If you choose to do a project that is primary about mathematical theory or is otherwise not the kind of thing for which you would produce figures and tables, please contact me and we'll discuss how to adjust this requirement. 
    \end{itemize}
    \item A \textbf{Conclusion} or \textbf{Discussion} section which discusses your findings in the context of the scientific question that you posed, and assesses the extent to which that question was or was not answered by your analysis.
    You should also describe some possible directions for future work.  
    \item A \textbf{Group Contributions Statement}. 
    Your group contributions statement should clearly describe the contributions of each group member to the project and the report. 
    One long paragraph (around two sentences per group member) should be sufficient. 
\end{enumerate}

Here is \href{https://www.pnas.org/doi/pdf/10.1073/pnas.1018985108}{an example} of a network science paper published in PNAS. 
Your report will look something like this one. 
Note that they placed their ``Materials and Methods'' section in a separate appendix section, whereas I expect it in the main text. 
This group of authors tends to make exceptionally fancy figures. 
Your figures don't have to be this fancy, but please make sure they are legible, clearly labeled, and convey your main point effectively. 


\pagebreak

\subsection*{Expectations for In-Class Presentation}

Your in-class presentation is your primary opportunity to share your learning with your peers. 
It will take place on either June 1st or June 3rd. 
Depending on the number of groups, presentations will be somewhere between 4 and 8 minutes in length.
The requirements are: 
\begin{enumerate}
    \item You use a \textbf{visual aid} to communicate your findings. In most cases, this should be slides. 
    Live demonstrations of coding projects may also be acceptable, but please ask if you're not sure. 
    \item Every group member makes an \textbf{approximately equal contribution} to the preparation and delivery of the presentation. 
\end{enumerate}













\end{document}