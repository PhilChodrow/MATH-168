\documentclass{hw}

\title{Short Essay 1}
\author{MATH 168, Spring 2022}
\date{Due Friday, April 8th}

\begin{document}


\section*{Short Essay 1}

This short essay is one of several that will help you prepare for your project in this class. 
It has two main aims: 
\begin{itemize}
    \item To help you practice producing mathematical documents using the \LaTeX{} typesetting language. 
    \item To help you begin thinking of networks that interest you and what kinds of questions you might wish to investigate about them. 
\end{itemize}

\subsection*{Prompt}

Find a \emph{scholarly source}, such as a published journal article, book, or preprint, that discusses a network that interests you. 
Any network is fine -- social, biological, information, physical, etc -- but it must be described in a scholarly source \textbf{\emph{other} than Newman's book}. 
Cite this source will addressing the following questions: 
\begin{itemize}
    \item Where did this network come from? How was the data for it collected? What is one way in which the data collection methodology could potentially have been flawed, relative to the intentions of the researchers who collected the data? 
    \item What is the main question that the researchers asked about the network? What did they do to answer this question? Describe at least one of their analyses using some mathematical symbols and at least one display equation (i.e. a centered equation on its own line). 
    It's fine to replicate symbols or equations from the source, with appropriate acknowledgment. 
    \item What are two questions that \emph{you} have that you feel would be interesting questions for future work? 
\end{itemize}

\subsection*{Requirements}

This short essay will be graded out of 5 points. 
An essay that meets the following requirements is very likely to receive full credit, while an essay on which one of the requirements is missing is very likely not to receive full credit. 


\begin{enumerate}
    \item Your essay should include discussions of each of the questions above. 
    \item Your essay should be written in the \LaTeX{} typesetting language. 
    \item Your essay should include multiple mathematical symbols and at least one display equation (like this one):
    \begin{align}
        \mathbf{L} = \mathbf{D} - \mathbf{A}
    \end{align}
    \item Your essay should be no shorter than 500 words and no longer than 1,000 words.
    \item Your essay should be organized into paragraphs, each of which discusses a separate and clearly indicated set of topics. 
    A valid and ``safe'' way to organize your essay is to have one paragraph corresponding to each of the three bullet points. 
    \item Your essay should be written in clear, scholarly English. Errors related to spelling and grammar are not an issue provided that your meaning is clear. 
    \item Your essay should include at least one \textbf{graphic with a caption} that is relevant to the network. 
    This could be a graphic from the paper, an image that helps the reader interpret the meaning of nodes or edges, or any other relevant graphic.
    Your caption must be original.  
    Make sure to briefly discuss this graphic in your written essay. 
    \item Your essay should be \textbf{on time}. 
    Late submissions will lose 1 point (out of 5) per day. 
\end{enumerate}


\subsection*{Getting Started with \LaTeX}

Maybe you haven't written documents with \LaTeX{} before. 
That's ok! 
The easiest way to get started is to make a free account at \url{https://www.overleaf.com}. 
Once you're there, make a new project using the ``Example Project'' template. 
Remove, modify, and add content as needed. 
All the features you'll need are already illustrated there. 
Here is a more comprehensive reference: 
\url{http://mirrors.rit.edu/CTAN/info/lshort/english/lshort.pdf}. 



\end{document}