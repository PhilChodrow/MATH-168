\documentclass{hw}

\title{Short Essay 2}
\author{MATH 168, Spring 2022}
\date{Due Friday, April 15th}

\begin{document}


\section*{Short Essay 2}

This short essay is a \textbf{pitch} for the group project that you'll complete by the end of the quarter. 
Your aim in this pitch is to: 
\begin{itemize}
    \item Develop an idea.
    \item Communicate this to your classmates, Grace, and myself. 
    \item Convince all of us that your project idea is interesting, challenging, and feasible. 
\end{itemize}
While the other short essays will be submitted using \LaTeX, \textbf{this short essay is submitted as a post on Campuswire}. 
Because Campuswire uses the \href{https://www.markdownguide.org/cheat-sheet/}{Markdown} system for formatting text, you should compose your essay in a Markdown editor that allows you to preview the appearance of your text. 
There are many good free Markdown editors, including several cloud-based solutions and many extensions for text editors. 

This is a pitch, not a proposal. 
It is your best guess at what you'd like to do your project not, \textbf{not a commitment}. 
You are not obligated to work on the topic that you describe in your pitch; indeed, I expect that most of you will actually decide to work on something else. 
That's ok! 
For now, please make your best guess. 


\subsection*{Prompt}

Your pitch should be guided by a \emph{scientific question about networks}. 
At this point, your question can be relatively vague, and you don't need to understand all the mathematical details. 
Here are a few examples of reasonable scientific questions: 
\begin{itemize}
    \item How reliable is the physical infrastructure network of the internet against failures? 
    \item Based on estimates for transmission of infectious disease, can masking prevent epidemic outbreaks? Can it mitigate outbreaks that are already occurring? 
    \item How could Twitter's recently announced Edit button help or hinder the spread of disinformation on that platform? 
\end{itemize}
\emph{Note}: I expect that most of you will want to do projects involving some data, and the prompt below reflects that. 
If you are interested in pitching a project that is primarily about mathematical theory or techniques, please send me an email. 


Think for a while about your scientific question. 
If you're not sure, I suggest reading Newman Chapters 2-5, and perhaps skimming other chapters that look interesting, to develop ideas. 
Once you have a pretty good view of your question, write an essay of between 500 and 1000 words. 
Your essay should precisely follow the structure described on the next page.  

\pagebreak

 \noindent \textbf{First paragraph}: 
    \begin{itemize}
        \item State your scientific question in \textbf{bold font}.
        \item Explain why that question is interesting or important. 
    \end{itemize}
     \textbf{Second paragraph}:
    \begin{itemize}
        \item Discuss \textbf{one scholarly article} that studied a question similar to yours. 
        \item Describe how they used network methods or models to address that question. 
        \item Describe briefly their conclusions. 
        \item Highlight how your scientific question differs from the ones in the scholarly source. In that context, describe one thing that you expect to \textbf{imitate} from the scholarly source, and one thing that you expect to do \textbf{differently}.
    \end{itemize}
    \textbf{Third paragraph}:
    \begin{itemize}
        \item Describe the network techniques that you plan to use to address your question. You are welcome to discuss techniques we talked about in class, but you are likely to need to consider ideas that we haven't discussed yet as well. See if you can find relevant chapters in Newman to describe some ideas, or do some research related to the tools used in your scholarly source from the previous paragraph. 
        \item Describe your \textbf{computational needs}, including data sets and programming languages. 
        \textbf{Link to an example of a data set} that you suggest using for the project.
        If you can't find one, feel free to ask us for help. 
        You can find some data sets at the repositories listed \href{http://www.philchodrow.com/intro-networks/appendices/additional_resources.html#data-sets}{here}. 
    \end{itemize}
    \textbf{Fourth paragraph}
    \begin{itemize}
        \item Describe at least one \textbf{risk} of your project. 
        What could potentially stop your project from being successful? 
        Examples: data not actually available, data is too big for the computation you need to do, etc. 
        \item On a more hopeful note, describe what you will have produced when your project is successful. 
        An algorithm? 
        A visualization? 
        New mathematical theory? 
    \end{itemize}

\pagebreak

\subsection*{To Submit Your Essay}

Format your essay using Markdown to make sure that it looks good. 
Then, post it in a \textbf{Note} (not a Question)  on Campuswire, and add the \textbf{Short Essay 2} tag. 
Give your post a descriptive title that captures some of the big ideas of what you want to do: ``Reliability of the internet against random failures,'' ``Impact of masking on epidemic severity,'' ``Impact of editing on Twitter disinformation spread,'' that kind of thing. 

Finally, \textbf{take a screencap of your post} and submit that screencap to Gradescope. 
It's ok if your screencap doesn't show your entire post; just make sure it shows your name, post title, and the first paragraph or so. 







\end{document}