\documentclass{hw}

\title{Assignment 1}
\author{MATH 168, Spring 2022}
\date{Due Monday, April 11th}

\begin{document}


\problem{}

Complete Exercise 6.1 in Newman. 

\problem{}

Complete Exercise 6.4 in Newman. 

\problem{}

The \emph{node-edge incidence matrix} of a graph with $n$ nodes and $m$ edges is a matrix $\mathbf{B} \in \mathbb{R}^{2m\times n}$. 
There are two rows of $\mathbf{B}$ for each edge $e$. 
If $e$ links nodes $j$ and $\ell$, then there is a row for the $j\rightarrow \ell$ ``direction'' and a row for the $\ell \rightarrow j$ ``direction''. 
So, we can write an individual entry of $\mathbf{B}$ as $b_{(j\rightarrow \ell), i}$. 
These entries are given by: 
\begin{align*}
    b_{(j \rightarrow \ell), i} = \begin{cases}
        -1 &\quad i = j \\ 
        +1 &\quad i = \ell \\ 
        0 &\quad \text{otherwise.}
    \end{cases}
\end{align*}

\part 

The Laplacian matrix $\mathbf{L}$ of a graph is defined in eq. (6.29) of Newman. 
Prove using direct matrix multiplication that $\mathbf{L}$ can be computed using one of the two formulae below (and figure out which one): 
\begin{align*}
    \mathbf{L} = \frac{1}{2}\mathbf{B}^T\mathbf{B} \quad \text{or} \quad \mathbf{L} = \frac{1}{2}\mathbf{B}\mathbf{B}^T\;. 
\end{align*}

\part 

Use your result from Part (a) to give a very short proof that $\mathbf{L}$ is a positive-semidefinite matrix. 

\problem{}

In section 6.14.1, Newman considers the role of the Laplacian $\mathbf{L}$ in partitioning or ``cutting'' graphs into groups.
Let's focus on the two-group case.
In eq. (6.37), Newman defines an objective function 
\begin{align}
    R(\mathbf{s}) = \frac{1}{4}\mathbf{s}^T\mathbf{L}\mathbf{s}\;,
\end{align}
where $\mathbf{s}\in \mathbb{R}^n$ is the vector with entries 
\begin{align}
    s_i = \begin{cases}
        +1 &\quad \text{node } i \text{ is in group 1,} \\ 
        -1 &\quad \text{node } i \text{ is in group 2.}
    \end{cases}
\end{align}
The idea is that a choice of $\mathbf{s}$ corresponds to a choice of groups for the nodes. 
Newman writes---somewhat uncarefully---that ``... our goal is to find the vector $\mathbf{s}$ that minimizes the cut size (6.37) for given $\mathbf{L}$.''

Assume throughout this problem that we are considering the Laplacian matrix $\mathbf{L}$ of a connected graph. 

\part 

Find the vector $\mathbf{s}$ that minimizes $R(\mathbf{s})$. 


\begin{hint}
There are multiple ways to do this, but carefully reading Chapter 6, section 14 of Newman is one. 
\end{hint}

\part 

Comment briefly (2-3 sentences is fine) on whether this vector is useful in the context of the graph partitioning problem. 

\part 

Suggest a \textbf{heuristic} idea for how you might modify Newman's framework in order to generate more useful solutions. 
What kinds of condition could you impose on $\mathbf{s}$ in order to make it more useful for the partitioning problem?

Please \textbf{draw} a simple network to accompany your discussion, and show how your proposed modification might partition the network. 
\textbf{You are not responsible for proving or exactly calculating anything for this part}.

\problem{}

In an upcoming part of the course, we will spend a lot of time working with \emph{random graphs}. 
In this problem, you will begin an analysis of the famous Erd\H{o}s-R\'enyi random graph model, which was first studied by Solomonoff and Rapoport. 

A random graph model is a \emph{probability distribution over graphs}. 
Many random graph models are specified as recipes for randomly generating a graph. 
Here's the recipe for the Erd\H{o}s-R\'enyi model $G(n,p)$: 
\begin{itemize}
    \item Start with $n$ nodes and no edges. 
    \item Between each pair of distinct nodes, place an edge with probability $p$. Each edge placement event is independent from each other edge placement event. 
\end{itemize}

Since $G(n,p)$ is a random graph, many of its properties can be described as random variables. 
Here are a few random variables:
\begin{itemize}
    \item $M$, the total number of edges in the graph. 
    \item $K_i$, the degree of node $i$. 
    \item The number of components in the graph. 
\end{itemize}
In this problem, you'll show a few things about some of these random variables. 

\part 

Consider the degree $K_i$ of node $i$. 
Argue that, if $p = c/n$ for some constant $c > 0$, then the degree of node $i$ is approximately distributed according to a Poisson distribution \textbf{when $n$ is very large} (i.e. $n\rightarrow \infty$). 
Carefully justify each step.  

\begin{hint}
    Homework 0. 
\end{hint}

\part 

Using your result from Part (a), compute the approximate mean and variance of $K_i$ in terms of $c$, again in the limit $n\rightarrow \infty$. 

We often report an estimate of a mean in the form $\mu \pm \sigma$, where $\sigma$ is the standard deviation. 
Calculate the standard deviation $\sigma$. 
In a large Erd\H{o}s-R\'enyi random graph in which each node has expected degree $100$, what is the estimate $100 \pm \sigma$? 

\begin{note}
    This is usually considered to be a small amount of variation in the node degrees when compared against real-world network data.  
\end{note}

\part 

Using your result from Part (a), compute the approximate probability mass function of $M$, the total number of edges in the graph. 
Find the approximate mean and variance of $M$. 
Here, you should assume that $n$ is large but finite. 

\problem{}

The \emph{degree distribution} of a random graph is an empirical probability distribution $p$ such that $p_k$ is the proportion of nodes with degree $k$. 
It is possible to prove that the degree distribution of an Erd\H{o}s-R\'enyi random graph is approximately the same as the Poisson distribution for an individual node degree. 

Write a computer program to produce two annotated plots. 

\begin{itemize}
    \item Generate an Erd\H{o}s-R\'enyi random graph with $10^6$ nodes and expected degree $100$. 
    Compute its degree distribution. 
    Plot this distribution. 
    Also plot (in a different color or symbol) a Poisson distribution with mean $100$. 
    Comment briefly on your findings. 
    \item Acquire, in any way you choose, a network data set containing at least $10^3$ nodes. 
    Some data sets are available at the sources listed \href{https://www.philchodrow.com/intro-networks/appendices/additional_resources.html#data-sets}{here}, but you're free to look anywhere else as well. 
    Compute and plot the degree distribution of your network. 
    Again plot a Poisson distribution with mean equal to the mean degree of the network. 
    Comment briefly on your findings. 
\end{itemize}

\begin{note}
    You may find it helpful to plot the degree distributions on log-log axes. 
\end{note}

\end{document}