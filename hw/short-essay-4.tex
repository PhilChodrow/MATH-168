\documentclass{hw}

\title{Short Essay 3}
\author{MATH 168, Spring 2022}
\date{Due Friday, April 29th}

\begin{document}

\section*{Short Essay 4}

Short Essay 4 is your \textbf{project proposal}. 
You should write it in collaboration with your group, using the \LaTeX{} document preparation system. 
\href{https://www.overleaf.com}{Overleaf} is a pretty good online resource for working in \LaTeX, but its free tier plan is limited to only two collaborators at a time. 
If you want, I can host an upgraded Overleaf project for you (at no additional cost to you); just email me with names and email addresses of your collaborators.  

Another option is to \LaTeX{} locally and use a file sync service such as Box or Dropbox. 
This approach requires you to choose a text editor for \LaTeX{}. 
There are many good options. 
A beginner-friendly option is \href{https://www.tug.org/texworks/}{TeXworks}. 
My personal favorite is \href{https://code.visualstudio.com}{Visual Studio Code} with the LaTeX Workshop extension. 

\section*{Expectations}

Your project proposal aims to: 
\begin{enumerate}
    \item Describe the scientific question you aim to address, why it is important, and to whom. 
    \item Describe the methods you will use to address your scientific question, including how you will access any data needed.
    \item Convince your reader that your idea is feasible and has a high chance of success. 
    \begin{itemize}
        \item Here, ``success'' means ``producing something interesting and insightful,'' which may be different from what you were originally aiming for. 
    \end{itemize} 
\end{enumerate}
Your proposal should achieve these goals in \textbf{no fewer than 1,000 words} and \textbf{no more than 1,500 words}. 

\subsection*{Structure}

Your proposal should include the following components, in order:
\begin{enumerate}
    \item Your project \textbf{title} and \textbf{names of collaborators}. 
    Please also include your group name! 
    BruinLearn requires me to give your group a name in the gradebook, so, uh, make it fun! 
    \item Your \textbf{Overview} should describe your scientific question, why it is important, and to whom it is important. 
    You should also describe the method(s) you will use to address your question. 
    Include brief discussion of at least \textbf{5 scholarly sources} related to either your question or your planned methods, and cite them using the \href{https://www.overleaf.com/learn/latex/Bibliography_management_with_bibtex}{Bib\TeX} citation management system. 
    \item Your \textbf{Resources Needed} section should describe anything you will need in order to complete the project. 
    An especially relevant resource needed is \textbf{data}. 
    If your project is going to use data, you are expected to find a data source, confirm that it has the fields you need for your idea, and link to it in this section. 
    Proposals that intend to use data but do not contain a link to the data needed will not receive full credit. 
    Other resources needed might include compute time or code packages. 
    I welcome you to ask for my help accessing the resources you need to complete your proposal! 
    \item Your \textbf{Tentative Workplan} should describe the major steps of your project. 
    What do you actually need to \emph{do}? 
    Examples of steps include: \emph{clean the data}, \emph{develop the model}, \emph{write the simulation code}, etc. 
    State which steps of your project you expect to be complete after 2 weeks, 4 weeks, and 6 weeks (the end of the 6th week after your proposal is due is the project due date). 
    \item Your \textbf{Vision and Contingencies} section should describe how you envision the final outcome of your project. 
    If everything goes perfectly, what will you have produced? 
    A cool model, an impressive data analysis, a surprising visualization? 
    This is your \emph{Full Success} scenario. 
    After you describe your Full Success scenario, also describe a \emph{Partial Success} scenario. 
    What's something that's likely to go wrong? 
    If it does go wrong, how will you shift course in order to still create and learn something valuable? 
    \item Conclude with an  \textbf{Anticipated Learnings} section. 
    What do you, as a group, expect to learn from working on this project? 
    Do you expect to be proud of what you learn? 
\end{enumerate}






\end{document}